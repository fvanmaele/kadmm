%% LyX 2.3.6.1 created this file.  For more info, see http://www.lyx.org/.
%% Do not edit unless you really know what you are doing.
\documentclass[10pt,twocolumn,ngerman,english]{article}
\usepackage{mathpazo}
\usepackage[scaled=0.95]{helvet}
\usepackage{courier}
\usepackage[T1]{fontenc}
\usepackage[latin9]{inputenc}
\usepackage{geometry}
\geometry{verbose,tmargin=2cm,bmargin=2cm,lmargin=2cm,rmargin=2cm}
\usepackage{array}
\usepackage{float}
\usepackage{calc}
\usepackage{url}
\usepackage{multirow}
\usepackage{amsmath}
\usepackage{amsthm}
\usepackage{amssymb}
\usepackage{graphicx}

\makeatletter

%%%%%%%%%%%%%%%%%%%%%%%%%%%%%% LyX specific LaTeX commands.
%% Because html converters don't know tabularnewline
\providecommand{\tabularnewline}{\\}
\floatstyle{ruled}
\newfloat{algorithm}{tbp}{loa}
\providecommand{\algorithmname}{Algorithm}
\floatname{algorithm}{\protect\algorithmname}

%%%%%%%%%%%%%%%%%%%%%%%%%%%%%% Textclass specific LaTeX commands.
\theoremstyle{plain}
\newtheorem{thm}{\protect\theoremname}
\theoremstyle{remark}
\newtheorem{rem}[thm]{\protect\remarkname}
\theoremstyle{definition}
\newtheorem{defn}[thm]{\protect\definitionname}
\theoremstyle{plain}
\newtheorem{prop}[thm]{\protect\propositionname}

%%%%%%%%%%%%%%%%%%%%%%%%%%%%%% User specified LaTeX commands.
\DeclareMathOperator{\argmin}{argmin}
\DeclareMathOperator{\prox}{prox}

\makeatother

\usepackage{babel}
\addto\captionsenglish{\renewcommand{\definitionname}{Definition}}
\addto\captionsenglish{\renewcommand{\propositionname}{Proposition}}
\addto\captionsenglish{\renewcommand{\remarkname}{Remark}}
\addto\captionsenglish{\renewcommand{\theoremname}{Theorem}}
\addto\captionsngerman{\renewcommand{\algorithmname}{Algorithmus}}
\addto\captionsngerman{\renewcommand{\definitionname}{Definition}}
\addto\captionsngerman{\renewcommand{\propositionname}{Satz}}
\addto\captionsngerman{\renewcommand{\remarkname}{Bemerkung}}
\addto\captionsngerman{\renewcommand{\theoremname}{Theorem}}
\providecommand{\definitionname}{Definition}
\providecommand{\propositionname}{Proposition}
\providecommand{\remarkname}{Remark}
\providecommand{\theoremname}{Theorem}

\begin{document}
\title{Total Variation Restoration of Spatially Variant Blur}
\author{Additional Project for ``Mathematical Image Processing'', SS 2021\\[0.5cm]Ferdinand
Vanmaele\\
\foreignlanguage{ngerman}{ }\texttt{\small{}Vanmaele@stud.uni-heidelberg.de}{\small{} }}
\maketitle
\begin{abstract}
In this project we try to deblur images where the blur is spatially
variant - that is, depending on the position of the image - and do
so in a computationally effective way. In the lecture, image deblurring
was formulated as a convex optimization problem, and discretized with
the \emph{discrete convolution}, a linear operator $A$. If the blur
is spatially \emph{invariant}, the discrete convolution $A$ is represented
by a Toeplitz or block circulant matrix, and we can use the Discrete
Fourier Transform (DFT) for efficient matrix operations. If the blur
is spatially \emph{variant}, $A$ is non-Toeplitz in general.

In the papers \cite{Yun-2017,Afonso-2011}, the authors focus on the
\emph{alternate direction method of multipliers} (ADMM) to solve this
problem. To reduce the problem size, the paper \cite{Yun-2017} finds
an approximate solution in a low-dimensional \emph{Krylov subspace.}
We verify their implementation and introduce the necessary theoretical
background. 
\end{abstract}

\section{Motivation and Overview\label{sec:Motivation-and-Overview}}

In image deblurring, we seek to recover the original, sharp image
by using a mathematical model of the blurring process. This process
is assumed to be \emph{linear}: if $f_{1}$ and $f_{2}$ are the blurred
versions of the exact images $u_{1}$ and $u_{2}$, then $f=\alpha f_{1}+\beta f_{2}$
is the blurred image of $u=\alpha u_{1}+\beta u_{2}$. Linear models
can describe several physical mechanisms such as relative motion between
the camera and subject (motion blur), bad focusing (defocusing blur),
or a number of other mechanisms \cite{Afonso-2011}.

The continous image acquisition model is, in particular, given by
the integral: \cite{Nagy-1998}
\begin{equation}
f(s)=\int_{\mathbb{R}^{2}}p(s,t)u(t)\,\text{d}t,\quad s\in Q\label{eq:continuous-model-1}
\end{equation}
where $p$ describes the PSF (\emph{point-spread function}), $u$
the exact image, $f$ the observed (blurred) image, and $Q\subset\mathbb{R}^{2}$
the \emph{field of view}. In the usual applications in astronomy,
microscopy and photographic imaging, the PSF has finite width \cite{Calvetti-2005},
i.e. 
\[
p(s,t)\approx0\quad\text{for}\quad\|s-t\|>\delta,\ \delta>0.
\]
Thus, we can restrict the integral to the \emph{extended field of
view} $B=Q+D_{\delta}$, where $D_{\delta}$ denotes a disc of radius
$\delta$ centered at the origin. (Compare Figure \ref{fig:The-PSF-spills})
 We now divide $B$ into $n_{B}$ identical pixels of area $h$ (assumed
$\equiv1$), with $n$ pixels inside the field of view:
\begin{equation}
f(s_{l})=\sum_{k=1}^{n_{B}}p(s_{l},t_{k})u(t_{k}),\qquad1\leq l\leq n\label{eq:discrete-model-1}
\end{equation}
We limit ourselves to estimating the image $u$ from the data within
the field of view $Q$. The observed image contains no information
of the true scenery outside of $Q$. We thus use one of the following
(artificial) boundary conditions: \cite[4.1 Basic Structures]{Hansen-2006}

\begin{itemize}
\item The \textbf{zero boundary condition} is a good choice when the exact
image is mostly zero outside the boundary\textemdash as is the case
for many astronomical images with a black background. Unfortunately,
the zero boundary condition has a bad effect on reconstructions of
images that are nonzero outside the border.
\item The \textbf{periodic boundary condition} is frequently used in image
processing. This implies that the image repeats itself (endlessly)
in all directions.
\item In some applications it is reasonable to use a \textbf{reflexive boundary
condition}, which implies that the scene outside the image boundaries
is a mirror image of the scene inside the image boundaries.
\end{itemize}
\begin{rem}
When the blur introduced by the image acquisition process is \emph{spatially
invariant}, the PSF $p$ does not depend on the spatial location $s$,
and can be specified by:
\begin{equation}
p(s,t)=p(s-t)\label{eq:psf-invariant}
\end{equation}
and the sum becomes the two-dimensional convolution operation \cite[4.1]{Yun-2017}.
\end{rem}

\begin{rem}
In this project we assume the point-spread function is a \emph{known
}quantity, for example by deriving experimentally from point sources
(such as fluorescent beads in microscopy \cite{Stuurman-2012}). If
the blurring model is not known, or not with sufficient accuracy,
we speak of \emph{blind deblurring} \cite{Almeida-2010}.
\end{rem}


\subsection{Matrix formulation\label{subsec:Matrix-formulation}}

We can arrange the elements of the images $f$ and $u$ into column
vectors\emph{ $\text{vec}(f)$} and $\text{vec}(u)$, by stacking
an $M\times N$ image into an $n=MN$ vector. The equation (\ref{eq:discrete-model-1})
can then be written as a linear system:
\begin{equation}
\text{vec}(f)=A\,\text{vec}(u).\label{eq:discrete-model-2}
\end{equation}
For images of sizes $M\times N$, the sizes of vectors are $MN\times1$,
and the size of the matrix $A$ is $MN\times MN$. The construction
of this matrix is described in detail in \cite[4. Structured Matrix Computations]{Hansen-2006}.

 If $p$ describes spatially invariant blur, the matrix $A$ has
a specific structure depending on the boundary conditions described
above. In particular, $A$ is \emph{BCCB} (block circulant with circulant
blocks) for periodic boundary conditions, and \emph{Toeplitz-plus-Hankel}
for reflexive boundary conditions. Matrix computations can then be
implemented efficiently with the Discrete Fourier transform (DFT)
and Discrete Cosine Transform (DCT), respectively \cite[4.2.2, 4.3]{Hansen-2006}. 

With \emph{spatially variant} blur, this is no longer the case. In
this case, there is not a single PSF that can be used to represent
the blurring operation; point sources in different locations of the
image may result in different PSFs. In the most general case, there
is a different PSF for each pixel in the image. Here we consider situations
where there is a continuous and smooth variation in the PSF with respect
to the position of the point source.

\begin{figure}[b]
\begin{centering}
\includegraphics[width=0.6\columnwidth]{psf_image_boundary}
\par\end{centering}
\caption{The PSF ``spills over the edge'' at the image boundary (line), so
that the scene values outside the image affect what is recorded. \label{fig:The-PSF-spills}}
\end{figure}


\section{Problem Solving Approach\label{sec:Problem-Solving-Approach}}

For most scenarios of practical interest, the matrix $A$ formulated
in \ref{subsec:Matrix-formulation} has a very high condition number,
with singular values decaying to, and clustering at 0 \cite[1.4]{Hansen-2006}.
This leads to the following problems:
\begin{enumerate}
\item For the system $Ax=f$, noise in the data term $f$ is amplified by
the inversion of very small singular values of $A$ (Figure \ref{fig:noise-inverse-filtering}). 
\item Small perturbations in the matrix $A$ (e.g. from differences between
the estimated and true PSF) may lead to poor deblurring results.
\end{enumerate}
\begin{figure}[h]
\begin{centering}
\includegraphics[width=0.8\columnwidth]{inverse_filtering}
\par\end{centering}
\caption{(a) Original image; (b) noise-free image, corrupted by linear motion
blur; (c) solution obtained by inverse filtering in the noise-free
case; (d) solution obtained by inverse filtering in the noisy case
(white Gaussian noise added to the image in (b))\label{fig:noise-inverse-filtering}
}
\end{figure}

In this project, we approach the first issue by adding \emph{regularizers},
in a constrained optimization problem of the form:
\begin{equation}
\min_{u\in\mathbb{R}^{n}}\phi(u)\quad\text{subject to}\quad\|Au-f\|_{2}\leq\varepsilon\label{eq:constrained-1}
\end{equation}
where $\phi:\mathbb{R}^{n}\rightarrow\overline{\mathbb{R}}=\mathbb{R}\cup\{-\infty,+\infty\}$
is a regularization function, and $\varepsilon\geq0$ a parameter
which depends upon the noise variance. This can be written as the
unconstrained problem:
\begin{equation}
\min_{u\in\mathbb{R}^{n}}\phi(u)+\delta_{B(\varepsilon,f)}(Au)\label{eq:unconstrained-1}
\end{equation}
where $\delta$ is the indicator function on the closed ball $B(\varepsilon,y)$
of radius $\varepsilon$, centered at $y$. To limit the dynamic range
of the solution, we add the constraint $0\leq x_{i}\leq1$ for $i\in[n]$
\cite[4.2]{Yun-2017}. This results in the unconstrained problem:
\begin{align}
\min_{u\in\mathbb{R}^{n}}J(u) & :=\min_{u\in\mathbb{R}^{n}}\phi(u)+\delta_{B(\varepsilon,f)}(Au)+\delta_{[0,1]^{n}}(u)\label{eq:unconstrained-2}
\end{align}

Popular choices for the regularizer are $\phi(u)=\|u\|_{1}$, the
$\ell_{1}$ norm, and 
\[
\phi(u)=TV(u)=\sum_{i=1}^{n}|u_{i}-u_{i-1}|,
\]
the 1-dimensional discrete total variation.
\begin{rem}[Existence of solution]
If $\phi$ is proper, convex and lower semi-continuous (such as TV),
then the functional $J$ is proper, convex and lower semi-continuous
(note that $B(\epsilon,f)$ is non-empty). $J$ is also coercive,
because $\delta_{[0,1]^{n}}$ is coercive by boundedness of the set
$[0,1]^{n}$. It follows that problem (\ref{eq:unconstrained-2})
has a solution. 
\end{rem}

\begin{rem}
Restoration algorithms also consider the unconstrained problem:
\begin{equation}
\min_{u\in\mathbb{R}^{n}}\frac{1}{2}\|Au-f\|_{2}^{2}+\tau\phi(u)\label{eq:alternative_problem}
\end{equation}
The formulation (\ref{eq:constrained-1}) has the benefit that the
parameter $\varepsilon$ has a clear meaning, i.e. it is proportional
to the noise standard deviation \cite[I-D]{Afonso-2011}.
\end{rem}


\subsection{Algorithms\label{subsec:Algorithms}}

If the regularizer is convex, problem (\ref{eq:constrained-1}) is
convex, but the very high dimensions (often $\gg10^{6})$ of $f$,
and possibly $u$, preclude direct application of off-the-shelf optimization
algorithms. Efficient restoration algorithms, such as those described
in \cite{Afonso-2011}, make assumptions on the structure of $A$. 

Specifically, spatially invariant blur with a periodic boundary condition
results in a block circulant matrix. This matrix has a spectral decomposition
$A=F^{\ast}\Lambda F$ given by the DFT matrix $F$ (which, assuming
$n=2^{k}$, can be computed in $\mathcal{O}(n\log_{2}n)$ time, see
\cite[4.2.1]{Hansen-2006}.) This in turn allows many matrix computions
to be performed efficiently. 

In the case of spatially variant blur (that is, the PSF changes depending
on the location of pixels in an observed image), the matrix $A$ is
no longer block-circulant, Toeplitz, or even symmetric. We may instead
attempt to find an approximate solution in a $m$-dimensional subspace
of $\mathbb{R}^{n}$, with $m\ll n$. The paper \cite{Yun-2017} uses
this approach to construct a \emph{Krylov subspace}, and formulates
the minimization problem inside this subspace to reduce computational
complexity.

In both papers \cite{Yun-2017,Afonso-2011}, the focus is on ADMM
(or \emph{alternate direction method of multipliers) }to solve the
minimization problem. ADMM is described in \cite{Boyd-2010} as ``an
algorithm that is intended to blend the decomposability of dual ascent
with the superior convergence properties of the method of multipliers'',
and has a close relationship with Douglas-Rachford iterative schemes.
In section \ref{sec:Theoretical-Properties}, we describe these methods
in detail. 

\section{Theoretical Properties\label{sec:Theoretical-Properties}}

\subsection{Alternate method of multipliers\label{subsec:Alternate-method-of-multipliers}}

The problem (\ref{eq:unconstrained-1}) can be seen as an unconstrained
optimization problem of the form 
\begin{equation}
\min_{u\in\mathbb{R}^{n}}f_{1}(u)+f_{2}(Gu)\label{eq:split-problem-1}
\end{equation}
where $G\in\mathbb{R}^{d\times n}$, $f_{1}:\mathbb{R}^{n}\rightarrow\overline{\mathbb{R}}$,
and $f_{2}:\mathbb{R}^{d}\rightarrow\overline{\mathbb{R}}$. \medskip{}

\emph{Variable splitting} is a simple procedure that consists in creating
a new variable, say $v$, to serve as the argument of $f_{2}$, under
the constraints that $v=Gu$: \cite[II.A]{Afonso-2011}
\begin{equation}
\min_{u\in\mathbb{R}^{n},\,v\in\mathbb{R}^{d}}f_{1}(u)+f_{2}(v)\quad\text{subject to}\quad Gu-v=0.\label{eq:split-problem-2}
\end{equation}
The rationale behind variable splitting is that it may be easier to
solve the constrained problem (\ref{eq:split-problem-2}) than to
solve its equivalent unconstrained counterpart (\ref{eq:split-problem-1})
We now form the \emph{augmented Lagrangian} for this problem:
\[
L_{\mu}(u,v,\lambda)=f_{1}(u)+f_{2}(v)+\lambda^{T}(Au-v)+\frac{\mu}{2}\|Gu-v\|_{2}^{2}
\]
Note that if $(u,v)$ is a feasible solution, then 
\[
L_{\mu}(u,v,\lambda)=f_{1}(u)+f_{2}(v).
\]
ADMM minimizes $L_{\mu}$ through the iterations \cite[3.1]{Boyd-2010}
\begin{align}
u_{k+1} & \in\argmin_{u}L_{\mu}(u,v_{k},\lambda_{k}),\nonumber \\
v_{k+1} & \in\argmin_{v}L_{\mu}(u_{k+1},v,\lambda_{k}),\label{eq:admm-iteration}\\
\lambda_{k+1} & \in\lambda^{k}+\mu(Au_{k+1}-v_{k+1}).\nonumber 
\end{align}
By setting the residual $r:=Gu-v$ and the \emph{scaled dual variable
}$d:=-\frac{1}{\mu}\lambda$, we get
\begin{align*}
\lambda^{T}r+\frac{\mu}{2}\|r\|_{2}^{2} & =(\mu/2)\|r+(\mu/2)\lambda\|_{2}^{2}-(1/2\mu)\|\lambda\|_{2}^{2}\\
 & =(\mu/2)\|r-d\|_{2}^{2}-(\mu/2)\|d\|_{2}^{2}.
\end{align*}
The removal of constant factors then results in the \emph{scaled formulation}
\begin{align}
u_{k+1}\in & \argmin_{u}f_{1}(u)+\frac{\mu}{2}\|Gu-v_{k}-d_{k}\|_{2}^{2},\nonumber \\
v_{k+1}\in & \argmin_{v}f_{2}(v)+\frac{\mu}{2}\|Gu_{k+1}-v-d_{k}\|_{2}^{2},\label{eq:admm-scaled}\\
d_{k+1}= & d_{k}-(Gu_{k+1}-v_{k+1}).\nonumber 
\end{align}
If $f_{2}$ is proper, convex and lsc, then the $v^{k}$ update uniquely
exists, and can be written using the \emph{proximal operator}
\[
v_{k+1}=\prox_{f_{2}/\mu}(Gu_{k+1}-d_{k}).
\]
For problem (\ref{eq:unconstrained-2}), we describe the proximal
operators in Section \ref{subsec:Proximal-operators}. Because it
contains more than 2 summands, we generalize ADMM in Section \ref{subsec:Generalized-ADMM}
to handle $J$ summands.
\begin{rem}[Convergence]
Under the assumptions that $f_{1}$ and $f_{2}$ are both closed,
proper and convex, that $A$ has full column rank, and that both subproblems
are solved exactly, then ADMM is shown to converge. As long as the
sequence of errors is absolutely summable, convergence is not compromised
\cite[Theorem 1]{Afonso-2011}. The convergence rate is $\mathcal{O}(1/k)$,
that is, 
\[
L_{\mu}(u^{k})-L_{\mu}(u^{\ast})=\mathcal{O}(1/k)
\]
with $k$ the iteration number \cite{He-2012}. We do not know of
a convergence result for ADMM in $m$-dimensional (Krylov) subspaces.
\end{rem}


\subsection{Proximal operators\label{subsec:Proximal-operators}}

The proximal operator for $\delta(\cdot|S)$ is given as the projection
on the set $S$. Therefore:
\[
\prox_{\delta(\cdot|B(\epsilon,f))}(s)=\begin{cases}
f+\epsilon\frac{s-f}{\|s-f\|_{2}}, & \|s-f\|_{2}>\epsilon,\\
s, & \|s-f\|_{2}\leq\epsilon,
\end{cases}
\]
and, component-wise,
\[
\prox_{\delta(\cdot|[0,1]^{n}))}(s)_{i}=\begin{cases}
0, & s_{i}\leq0,\\
s_{i}, & 0<s_{i}<1,\\
\beta, & s_{i}\geq1.
\end{cases}
\]
The proximal operator for the 1-dimensional total variation can be
computed with e.g. Condat's taut string algorithm \cite{Condat-2013}.

\subsection{Generalized ADMM\label{subsec:Generalized-ADMM}}

To approach problem (\ref{eq:unconstrained-2}), and eliminate the
$f_{1}$ term in the $u$-update of (\ref{eq:admm-scaled}), we consider
a generalization of problem (\ref{eq:split-problem-1}). Instead of
two functions, we have $J$ functions, that is \cite[II.D]{Afonso-2011}
\begin{equation}
\min_{u\in\mathbb{R}^{d}}\sum_{j=1}^{J}g_{j}(G^{(j)}u),\label{eq:split-problem-3}
\end{equation}
where $g_{j}:\mathbb{R}^{p_{j}}\rightarrow\overline{\mathbb{R}}$
are closed, proper and convex functions, $u\in\mathbb{R}^{d}$, and
$G^{(j)}\in\mathbb{R}^{p_{j}\times d}$ are arbitrary matrices. 

The minimization problem (\ref{eq:split-problem-3}) can be written
as the problem of two summands (\ref{eq:split-problem-1}), using
the following correspondences:
\[
f_{1}=0,\quad G=\begin{bmatrix}G^{(1)}\\
\vdots\\
G^{(J)}
\end{bmatrix}\in\mathbb{R}^{p\times d}
\]
with $p=p_{1}+\cdots+p_{J}$, and $f_{2}:\mathbb{R}^{p\times d}\rightarrow\overline{\mathbb{R}}$
given by:
\[
f_{2}(v)=\sum g_{j}(v^{(j)}),\quad v^{(j)}\in\mathbb{R}^{p_{j}}
\]
and $v=[(v^{(1)})^{T},\dots,(v^{(J)})^{T}]^{T}$. We now apply ADMM
with
\[
d_{k}=\begin{bmatrix}d_{k}^{(1)}\\
\vdots\\
d_{k}^{(J)}
\end{bmatrix},\quad v_{k}=\begin{bmatrix}v_{k}^{(1)}\\
\vdots\\
v_{k}^{(J)}
\end{bmatrix}.
\]

Note that:
\begin{itemize}
\item $f_{1}=0$ turns the $u$-update in ADMM into a least-squares problem,
which has a unique solution if $A$ has full column rank.
\item The way of mapping problem (\ref{eq:split-problem-3}) into problem
(\ref{eq:split-problem-1}) allows decoupling the minimization step
in the $v$-update of ADMM into a set of $J$ independent ones. That
is,
\[
v_{k+1}=\argmin_{v}f_{2}(v)+\frac{\mu}{2}\|Gu_{k+1}-d_{k}\|_{2}^{2}
\]
can be written as
\begin{align*}
\begin{bmatrix}v_{k+1}^{(1)}\\
\vdots\\
v_{k+1}^{(J)}
\end{bmatrix}= & \arg\min_{v^{(1)},\dots,v^{(J)}}g_{1}(v^{(1)})+\cdots+g_{J}(v^{(J)})\\
 & +\frac{\mu}{2}\left\Vert \begin{bmatrix}G^{(1)}\\
\vdots\\
G^{(J)}
\end{bmatrix}u_{k+1}-\begin{bmatrix}v^{(1)}\\
\vdots\\
v^{(J)}
\end{bmatrix}-\begin{bmatrix}d_{k}^{(1)}\\
\vdots\\
d_{k}^{(J)}
\end{bmatrix}\right\Vert _{2}^{2}
\end{align*}
\end{itemize}
This is summarized in Algorithm \ref{alg:ADMM-2}. 

\begin{algorithm}
Choose $\mu>0$, $v_{0}^{(1)},\dots,v_{0}^{(J)},$ and $d_{0}^{(1)},\dots,d_{0}^{(J)}$.

\textbf{Repeat:}

\hspace{2em}$u_{k+1}\leftarrow\argmin_{u}\|Gu-v_{k}-d_{k}\|_{2}^{2}$

\hspace{2em}\textbf{For }$j=1,\dots,J$ \textbf{Do}:

\hspace{2em}\hspace{2em}$v_{k+1}^{(j)}\leftarrow\prox_{g_{j}/\mu}(G^{(j)}u_{k+1}-d_{k}^{(j)})$

\hspace{2em}\hspace{2em}$d_{k+1}^{(i)}\leftarrow d_{k}^{(i)}-(G^{(j)}u_{k+1}-v_{k+1}^{(j)})$

\hspace{2em}\textbf{EndDo}

\hspace{2em}$k\leftarrow k+1$

\textbf{Until} some stopping criterion is satisfied.

\caption{Generalized ADMM for problem (\ref{eq:split-problem-3}). \label{alg:ADMM-2}}

\end{algorithm}


\subsection{Krylov subspaces\label{subsec:Krylov-subspaces}}

In the optimization problem (\ref{eq:unconstrained-2}), the very
high dimension $n$ of the observed image $u$ precludes a direct
application of ADMM, as explained in Section \ref{subsec:Algorithms}.
We thus wish to apply optimization problems in a \emph{subspace} of
dimensionality $m\ll n$.
\begin{defn}[{\cite[6.2]{Saad-2003}}]
The \textbf{Krylov subspace} of order $m$, denoted by $\mathcal{K}_{m}$,
is the vector space spanned by $v\in\mathbb{R}^{n}$ and the first
$m-1$ powers of $A\in\mathbb{R}^{n\times n}$:
\[
\mathcal{K}_{m}(A,v)=\text{span}\{v,Av,A^{2}v,\dots,A^{m-1}v\}.
\]
\end{defn}

The Krylov subspaces form an increasing family of subspaces, with
dimension bounded by $n$. It is the subspace of all vectors in $\mathbb{R}^{n}$
which can be written as $x=p(A)v$, where $p$ is a polynomial of
degree not exceeding $m-1$. In practice, constructing the Krylov
subspaces amounts to determining their basis. The power basis $(v,Av,A^{2}v,\ldots,A^{m-1}v)$
converges to (a multiple of) an eigenvector that corresponds to the
largest eigenvalue of $A$, in absolute value. Therefore we construct
an orthonormal basis $\{v_{1},v_{2},\ldots,v_{m}\}$ as follows. 

\textbf{Arnoldi's method} applies Gram-Schmidt orthonormalization
procedure to the vectors obtained by successive products of the matrix
$A$. Algorithm \ref{alg:Arnoldi-Modified-Gram-Schmidt} \cite[6.3.2]{Saad-2003}
gives a practical implementation, using the modified Gram-Schmidt
procedure for improved numerical stability.  

\begin{algorithm}[h]
$r_{0}\leftarrow f-Au_{0}$

$q_{1}\leftarrow r_{0}/\|r\|_{2}$

\textbf{For} $i=1,2,\ldots,m$ \textbf{Do}:

\hspace{2em}$v\leftarrow Aq_{i}$

\hspace{2em}\textbf{For} $j=1,\cdots,i$ \textbf{Do}:

\hspace{2em}\hspace{2em}$h_{j,i}\leftarrow q_{j}^{T}v$

\hspace{2em}\hspace{2em}$v\leftarrow v-h_{j,i}q_{j}$

\hspace{2em}\textbf{EndDo}

\hspace{2em}$h_{i+1,i}\leftarrow\|v\|_{2}$. \textbf{If} $h_{i+1,i}=0$
\textbf{Stop}

\hspace{2em}$q_{i+1}\leftarrow v/h_{i+1,i}$

\textbf{EndDo}

\caption{Arnoldi($A,f,m,u_{0})$\label{alg:Arnoldi-Modified-Gram-Schmidt}}
\end{algorithm}
\medskip{}

\begin{prop}[{\emph{\cite[6.5]{Saad-2003}}}]
\emph{The following relation holds for the Arnoldi basis: 
\begin{equation}
AQ_{m}=Q_{m+1}H_{m},\label{eq:arnoldi-relations}
\end{equation}
where $Q_{m}$ denotes the $n\times m$ matrix with orthonormal column
vectors $v_{1},\ldots,v_{m}$, and $H_{m}$ the $(m+1)\times m$ upper
Hessenberg matrix with nonzero entries $h_{ij}$.}
\end{prop}

For a linear system $Au=f$, Krylov subspace methods extract an approximate
solution $u_{m}$ from the affine subspace $u_{0}+\mathcal{K}_{m}(A,r_{0})$
of dimension $m$, by imposing the condition $b-Au_{m}\bot\mathcal{L}_{m}$.
Here, $\mathcal{L}_{m}$ is another subspace of dimension $m$, $u_{0}$
represents an arbitrary initial guess to the solution, and $r_{0}=b-Au_{0}$
is the initial residual. 

For example, if $\mathcal{L}_{m}=A\mathcal{K}_{m}$, the method minimizes
the $2$-norm of the residual $b-Ax$ over $x\in x_{0}+\mathcal{K}_{m}$
\cite[Proposition 5.3]{Saad-2003}. Such a method (e.g. GMRES) can
work with general non-symmetric matrices $A$, provided they are non-singular.

We can now state the following proposition.
\begin{prop}[{\cite[Proposition 3]{Yun-2017}}]
\label{prop:krylov-least-squares}\emph{Under the assumption $u\in u_{0}+\mathcal{K}_{m}(A,r_{0})$,
the least squares problem on line 3 of Algorithm \ref{alg:ADMM-2}
is reduced to
\begin{equation}
\min_{\alpha\in\mathbb{R}^{m}}\left\Vert \begin{bmatrix}H_{m}\\
\overline{G}_{2}\\
\vdots\\
\overline{G}_{J}
\end{bmatrix}\alpha-\begin{bmatrix}Q_{m+1}^{T}(v_{1}+d_{1})\\
Q_{m}^{T}(v_{2}+d_{2})\\
\vdots\\
Q_{m}^{T}(v_{J}+d_{J})
\end{bmatrix}\right\Vert _{2}^{2}\label{eq:krylov-least-squares}
\end{equation}
where $\overline{G}_{j}:=Q_{m}^{T}G_{j}Q_{+m}\in\mathbb{R}^{m\times m}$
for $j=2,\dots,J$. }
\end{prop}

\begin{proof}
Follows from relation (\ref{eq:arnoldi-relations}) and orthogonality
of $Q_{m}$.
\end{proof}

\subsection{TV restoration in Krylov subspace\label{subsec:TV-restoration}}

We now combine the results of the previous sections. Consider again
the unconstrained problem (\ref{eq:unconstrained-2}) with $\phi=TV$:
\[
\min_{u\in\mathbb{R}^{n}}TV(u)+\delta_{B(\varepsilon,f)}(Au)+\delta_{[0,1]^{n}}(u)
\]

Formulated in Krylov subspace $\mathcal{K}_{m}(A,f)$, with initial
solution $u_{0}=0$ and matrix basis $Q_{m}$, the problem becomes:
\begin{equation}
\min_{\alpha\in\mathbb{R}^{m}}\delta_{B(\varepsilon,f)}(Q_{m+1}H_{m}\alpha)+TV(Q_{m}\alpha)+\delta_{[0,1]^{m}}(Q_{m}\alpha)\label{eq:krylov-problem}
\end{equation}
using the identity $AQ_{m}=Q_{m+1}H_{m}$ for the first summand. Algorithm
\ref{alg:ADMM-2} is applied to this problem by setting
\begin{align*}
G^{(1)} & =Q_{m+1}H_{m},\quad G^{(2)}=G^{(3)}=Q_{m},
\end{align*}
and applying Proposition \ref{prop:krylov-least-squares}. $Q_{m+1}$
and $H_{m}$ are constructed with Arnoldi (Algorithm \ref{alg:Arnoldi-Modified-Gram-Schmidt})
from the blur matrix $A\in\mathbb{R}^{MN\times MN}$. This results
in Algorithm \ref{alg:KADMM-TV}.

\begin{algorithm}
Choose $\mu>0$, $v_{1},\dots,v_{3}\in\mathbb{R}^{MN},$ $d_{1},\dots,d_{3}\in\mathbb{R}^{MN}$,
$k=0$. 

$Q_{m+1},H_{m}\leftarrow\text{Arnoldi}(A,f,m,0)$

\textbf{Repeat:}

\hspace{2em}$\alpha\leftarrow\argmin_{\alpha}\left\Vert \begin{bmatrix}H_{m}\\
I_{m}\\
I_{m}
\end{bmatrix}\alpha-\begin{bmatrix}Q_{m+1}^{T}(v_{1}+d_{1})\\
Q_{m}^{T}(v_{2}+d_{2})\\
Q_{m}^{T}(v_{3}+d_{3})
\end{bmatrix}\right\Vert _{2}^{2}$

\hspace{2em}$v_{1}\leftarrow\prox_{\delta(\cdot|B(\epsilon,f))}(Q_{m+1}H_{m}\alpha-d_{1})$

\hspace{2em}$d_{1}\leftarrow d_{k}^{(1)}-(Q_{m+1}H_{m}\alpha-v_{1})$

\hspace{2em}$v_{2}\leftarrow\prox_{TV/\mu}(Q_{m}\alpha-d_{2})$

\hspace{2em}$d_{2}\leftarrow d_{2}-(Q_{m}\alpha-v_{2})$

\hspace{2em}$v_{3}\leftarrow\prox_{\delta(\cdot|[0,1]^{n})}(Q_{m}\alpha-d_{3})$

\hspace{2em}$d_{3}\leftarrow d_{3}-(Q_{m}\alpha-v_{3})$

\textbf{Until} some stopping criterion is satisfied.

\caption{KADMM for TV restoration\label{alg:KADMM-TV}}

\end{algorithm}


\subsection{Complexity analysis\label{subsec:Complexity-analysis}}

Let $n=MN$ be the dimension of the observed image, and $m$ the order
of the Krylov subspace. Let $k$ be the number of iterations. We observe
the following for Algorithm \ref{alg:KADMM-TV}:
\begin{itemize}
\item Constructing the Arnoldi basis via Gram-Schmidt orthogonalization
takes $\mathcal{O}(m^{2}n)$ operations.
\item In each iteration, a least squares problem of size $(3m+1)\times m$
is solved, with $\mathcal{O}((3m+1)^{2}m)$ operations (using QR decomposition). 

This is much smaller than the least squares problem from a direct
application of ADMM, with size $3n\times n$ and complexity $\mathcal{O}(9n^{3})$.
\item The product $Q_{m}\alpha$ takes $mn$ operations. In each iteration,
the result can be stored and reused. 
\item The product $Q_{m+1}H_{m}\alpha$ requires $\mathcal{O}(2m^{2}+2mn)$
operations, compared to $\mathcal{O}(2mn+2n^{2})$ for the product
$AQ_{m}\alpha$.
\item The projection onto $B(\epsilon,f)$ and $[0,1]^{n}$ takes $\mathcal{O}(n)$
operations.
\item Condat's taut string algorithm has worst-case complexity $\mathcal{O}(n^{2})$,
with $\mathcal{O}(n)$ ``in all practical situations''.
\end{itemize}
In short, we have $\mathcal{O}(m^{2}n)$ for the Arnoldi basis, and
$\mathcal{O}(km^{3}+kn)$ for the ADMM iterations. Evidently, for
$k$ sufficiently large, the complexity of ADMM dominates. For $m\ll n$,
the algorithm then has complexity $\mathcal{O}(kn)$. For an experimental
verification, see Table \ref{tab:Runtimes}.

\begin{table}
\begin{centering}
{\small{}}%
\begin{tabular}{|c|c|c|c|}
\hline 
{\small{}$n$} & {\small{}Elapsed time} & {\small{}$n_{\max}/n$} & {\small{}$t_{\max}/t$}\tabularnewline
\hline 
{\small{}1617} & {\small{}0.0807s} & {\small{}95.4861} & {\small{}95.2307}\tabularnewline
\hline 
{\small{}9801} & {\small{}0.5342} & {\small{}15.7536} & {\small{}14.3783}\tabularnewline
\hline 
{\small{}38801} & {\small{}1.9732} & {\small{}3.9793} & {\small{}3.8928}\tabularnewline
\hline 
{\small{}87001} & {\small{}4.9271} & {\small{}1.7747} & {\small{}1.5590}\tabularnewline
\hline 
{\small{}154401} & {\small{}7.6811} & {\small{}1} & {\small{}1}\tabularnewline
\hline 
\end{tabular}\medskip{}
\par\end{centering}
\caption{Runtimes ($n$) for 100 ($k$) iterations, using parameters in Section
\ref{sec:Numerical-Methods}. The table shows the elapsed time to
be proportional to the problem size $n$.\label{tab:Runtimes}}
\end{table}


\section{Numerical Experiments\label{sec:Numerical-Methods}}

All methods were implemented using MATLAB on an Intel Core i5-2520
CPU. Images from the BSDS500 data set are blurred with spatially variant
blur (see below), with periodic boundary conditions. Gaussian noise
is added, with noise variance $\sigma^{2}$ determined by the BSNR
(\emph{blurred signal-to-noise ratio}), defined in \cite{Yun-2017}
as:
\[
\text{BSNR}=10\log\frac{\text{var}(Au)}{\sigma^{2}}\quad\Leftrightarrow\quad\frac{\text{var}(Au)}{\exp\left(\frac{1}{10}\text{BSNR}\right)}=\sigma^{2}
\]


\subsection{Spatially variant Gaussian blur\label{subsec:Spatially-variant-Gaussian}}

The spatially variant Gaussian blur model in \cite{Berisha-2014}
is used for the restoration experiments. The blur kernel at the pixel
location $s=(s_{1},s_{2})$ in the image acquisition model is obtained
from the two-dimensional separable Gaussian function
\begin{align*}
p(s,t) & =\frac{1}{\xi_{1}\xi_{2}}\exp\left(-\frac{1}{2}\frac{(t_{1}-s_{1})^{2}}{\sigma_{x}^{2}(s_{1})}+\frac{(t_{2}-s_{2})^{2}}{\sigma_{y}^{2}(s_{2})}\right),
\end{align*}
with normalizing constants 
\[
\xi_{1}=\sqrt{2\pi}\sigma_{x}(s_{1}),\quad\xi_{2}=\sqrt{2\pi}\sigma_{y}(s_{2}).
\]
The blur kernel changes depending on the location $x$, and hence
the blur is spatially variant.
\begin{rem}
$p(x,y)$ is seperable, that is, 
\[
p(s,t)=p_{1}(s_{1},t_{1})\,p_{2}(s_{2},t_{2})
\]
holds with
\begin{align*}
p_{1}(s_{1},t_{1}) & =\frac{1}{\xi_{1}}\exp\left\{ -\frac{1}{2}\frac{(t_{1}-s_{1})^{2}}{\sigma_{x}^{2}(s_{1})}\right\} \\
p_{2}(s_{2},t_{2}) & =\frac{1}{\xi_{2}}\exp\left\{ -\frac{1}{2}\frac{(t_{2}-s_{2})^{2}}{\sigma_{y}^{2}(s_{2})}\right\} 
\end{align*}
This means we can represent the matrix $A$ as a Kroneker product
$A_{1}\otimes A_{2}$, with $A_{1}$ and $A_{2}$ the 1-dimensional
blur matrices \cite[4.1.1]{Hansen-2006} for a given boundary condition
\cite[2.2.5]{Berisha-2014}.
\end{rem}

In \cite{Yun-2017}, three types of spatially variant blur are considered.
The first type has a smaller amount of blur at the center and a larger
amount of blur at the corners of an image. The variances are given
by:
\begin{align*}
\sigma_{x}(s_{1}) & =\gamma|0.5-s_{1}/M|+0.5,\\
\sigma_{y}(s_{2}) & =\gamma|0.5-s_{2}/N|+0.5.
\end{align*}
The second type is the opposite of the first case, with more severe
blur at the center and milder blur at the corners. The variances are
given by
\begin{align*}
\sigma_{x}(s_{1}) & =-\gamma|0.5-s_{1}/M|+2.5,\\
\sigma_{y}(s_{2}) & =-\gamma|0.5-s_{2}/N|+2.5.
\end{align*}
The third type has smaller amount of blur at the lower right corner
and larger amount of blur at the upper left part of an image. The
variances are given by
\begin{align*}
\sigma_{x}(s_{1}) & =\gamma(0.5-s_{1}/M)/2+1.25,\\
\sigma_{y}(s_{2}) & =\gamma(0.5-s_{2}/N)/2+1.25.
\end{align*}
The scalar $\gamma$ is set fixed to 4, and $21\times21$ kernel size
is used for discretization of $p$.

\subsection{Image restoration\label{subsec:Image-restoration}}

For the three types of blur described in \ref{subsec:Spatially-variant-Gaussian},
we applied KADDM to images with varying degrees of noise. To measure
the quality of the restoration, we used the mean squared error between
restored image $\hat{u}$ and original image $u$,
\[
\text{MSE}=\frac{1}{n}\sum_{i=1}^{n}(u_{i}-\hat{u}_{i}).
\]
Parameters for Algorithm \ref{alg:KADMM-TV} were chosen as follows.
\begin{itemize}
\item The initial values for $v_{i}$ and $d_{i}$ are sampled from the
uniform distribution $U[0,1]$. 
\item The order $m$ of the Krylov subspace is set to 16, to limit memory
requirements.
\item Relative change of residuals was used for the termination criterium,
i.e. $\|r^{i}-r^{i-1}\|/\|r\|<\text{TOL},$ with a tolerance of $10^{-6}$
and a maximum amount of iterations of 10,000.
\item The authors in \cite{Yun-2017} suggest $\epsilon=\sqrt{\tau MN}\sigma$
using Morozov's discrepancy principle, with $\tau=1$. Following \cite{Chuan-2014},
we used a fixed $\tau<1$ to avoid an oversmoothed image.
\end{itemize}
A comparison of degraded and restored images for each blur type is
given in Figures \ref{fig:blur-type-1}, \ref{fig:blur-type-2}, \ref{fig:blur-type-3}.
KADMM managed to deblur the images to a certain degree, with an MSE
of order of magnitude $10^{-3}$. Increasing the maximum iteration
count (to 100,000) did not lead to a significant decrease in MSE,
nor did setting $\epsilon$ to a low ($\approx$TOL) value; in the
presence of noise, latter lead to a ``speckle pattern'' in the restored
images.

\begin{figure}
\noindent\fbox{\begin{minipage}[t]{1\columnwidth - 2\fboxsep - 2\fboxrule}%
\begin{center}
\includegraphics[width=0.8\columnwidth]{02Deblurring/001_60db_corners_sv}
\par\end{center}%
\end{minipage}}\bigskip{}
\noindent\fbox{\begin{minipage}[t]{1\columnwidth - 2\fboxsep - 2\fboxrule}%
\begin{center}
\includegraphics[width=0.8\columnwidth]{02Deblurring/001_60db_corners_sv_restored}
\par\end{center}%
\end{minipage}}

\caption{Degraded (top) vs. restored (bottom) image after 7086 KADMM iterations
(blur type 1).\label{fig:blur-type-1}}
\end{figure}

\begin{figure}
\noindent\fbox{\begin{minipage}[t]{1\columnwidth - 2\fboxsep - 2\fboxrule}%
\begin{center}
\includegraphics[width=0.8\columnwidth]{02Deblurring/050_60db_center_sv}
\par\end{center}%
\end{minipage}}\bigskip{}
\noindent\fbox{\begin{minipage}[t]{1\columnwidth - 2\fboxsep - 2\fboxrule}%
\begin{center}
\includegraphics[width=0.8\columnwidth]{02Deblurring/050_60db_center_sv_restored}
\par\end{center}%
\end{minipage}}

\caption{Degraded (top) vs. restored (bottom) image after 4991 KADMM iterations
(blur type 2).\label{fig:blur-type-2}}
\end{figure}

\begin{figure}
\fbox{\begin{minipage}[t]{0.45\columnwidth}%
\begin{center}
\includegraphics[width=1\columnwidth]{02Deblurring/103_60db_gradient_sv}
\par\end{center}%
\end{minipage}}\hfill{}%
\fbox{\begin{minipage}[t]{0.45\columnwidth}%
\begin{center}
\includegraphics[width=1\columnwidth]{02Deblurring/103_60db_gradient_sv_restored}
\par\end{center}%
\end{minipage}}

\caption{Degraded (top) vs. restored (bottom) image after 4532 KADMM iterations
(blur type 3).\label{fig:blur-type-3}}
\end{figure}

\begin{table}[bh]
\begin{centering}
{\small{}}%
\begin{tabular}{|c|c|c|c|c|}
\hline 
{\small{}ID} & {\small{}Blur} & {\small{}Iterations} & {\small{}Time} & {\small{}MSE}\tabularnewline
\hline 
\multirow{3}{*}{{\small{}1}} & {\small{}1} & {\small{}7086} & {\small{}382.5s} & {\small{}$1.199\cdot10^{-3}$}\tabularnewline
\cline{2-5} \cline{3-5} \cline{4-5} \cline{5-5} 
 & {\small{}2} & {\small{}7171} & {\small{}395s} & {\small{}$5.23\cdot10^{-4}$}\tabularnewline
\cline{2-5} \cline{3-5} \cline{4-5} \cline{5-5} 
 & {\small{}3} & {\small{}5490} & {\small{}296s} & {\small{}$7.53\cdot10^{-4}$}\tabularnewline
\hline 
\multirow{3}{*}{{\small{}50}} & {\small{}1} & {\small{}5186} & {\small{}365s} & {\small{}$3.370\cdot10^{-3}$}\tabularnewline
\cline{2-5} \cline{3-5} \cline{4-5} \cline{5-5} 
 & {\small{}2} & {\small{}4991} & {\small{}347.5s} & {\small{}$2.594\cdot10^{-3}$}\tabularnewline
\cline{2-5} \cline{3-5} \cline{4-5} \cline{5-5} 
 & {\small{}3} & {\small{}3458} & {\small{}248s} & {\small{}$2.695\cdot10^{-3}$}\tabularnewline
\hline 
\multirow{3}{*}{{\small{}103}} & {\small{}1} & {\small{}5509} & {\small{}378s} & {\small{}$2.223\cdot10^{-3}$}\tabularnewline
\cline{2-5} \cline{3-5} \cline{4-5} \cline{5-5} 
 & {\small{}2} & {\small{}4895} & {\small{}326s} & {\small{}$2.492\cdot10^{-3}$}\tabularnewline
\cline{2-5} \cline{3-5} \cline{4-5} \cline{5-5} 
 & {\small{}3} & {\small{}4532} & {\small{}272s} & {\small{}$1.855\cdot10^{-3}$}\tabularnewline
\hline 
\end{tabular}\medskip{}
\par\end{centering}
\caption{Numerical results for images 1, 50 and 103 from BSDS500, with spatially
variant (types 1-3) and invariant (IV) blur, and Gaussian noise (60dB
BSNR).\label{tab:numerical-results}}
\end{table}


\section{Conclusion}

This project presents a convex optimization framework based on ADMM
and Krylov subspace methods. The proposed KADMM finds an inexact solution
by restricting the solution space to a Krylov subspace of small order.
The basis of the Krylov subspace is found by the Arnoldi process,
and ADMM is applied in the Krylov subspace. The computational complexity
of the algorithm is then linear in the image size and number of iterations.
Unlike restoration methods for spatially invariant blur, which rely
on Fourier transforms for computational efficiency, the method is
applicable to general, non-symmetric matrices $A$.

Experiments on restoration of spatially variant blurred and noised
images show that, assuming full knowledge of the blurring operator
and noise variance, the image can be restored to a certain degree.
Improved results can likely be achieved with a careful hyperparameter
study, e.g. on the regularization parameter $\mu$ for the TV proximal
operator \cite{Deledalle-2012}.
\begin{thebibliography}{10}
\bibitem{Yun-2017}J.D. Yun, S. Yang. \emph{ADMM in Krylov Subspace
and Its Application to Total Variation Restoration of Spatially Variant
Blur}, SIAM J. Imaging Sciences, Vol. 10, No. 2, pp. 484-507, 2017.

\bibitem{Afonso-2011}M.V. Afonso, J.M. Bioucas-Dias, M. A. T. Figueiredo.
\emph{An Augmented Lagrangian Approach to the Constrainted Optimization
Formulation of Imaging Inverse Problems}, IEEE Transactions on Image
Processing, Vol. 20, No. 3, March 2011.

\bibitem{Boyd-2010}S. Boyd, N. Parikh, E. Chu, B. Peleato, J. Eckstein.
\emph{Distributed Optimization and Statistical Learning via the Alternating
Direction Method of Multipliers}, Foundations and Trends in Machine
Learning, Vol. 3, No. 1, 2010.

\bibitem{Hansen-2006}P. C. Hansen, J. G. Nagy, D. P. O'Leary. \emph{Deblurring
Images \textendash{} Matrices, Spectra and Filtering}, SIAM, 2006.

\bibitem{Nagy-1998}J. G. Nagy, D. P. O'Leary. \emph{Restoring images
degraded by spatially variant blur}, SIAM J. Sci. Comput., Vol. 19,
No. 4, pp. 1063-1082, July 1998.

\bibitem{Saad-2003}Y. Saad. \emph{Iterative Methods for Sparse Linear
Systems}, SIAM, 2003.

\bibitem{Stuurman-2012}N. Stuurman. \emph{Measuring a Point Spread
Function}, iBiology, 2012. \url{https://www.ibiology.org/talks/measuring-a-point-spread-function/}

\bibitem{Almeida-2010}M.S.C. Almeida, L.B. Almeida. \emph{Blind and
Semi-Blind Deblurring of Natural Images}, IEEE Transactions on Image
Processing, Vol. 19, No. 1, January 2010.

\bibitem{Calvetti-2005}D. Calvetti, E. Somersalo. \emph{Bayesian
image deblurring and boundary effects}, Advanced Signal Processing
Algorithms, Architectures and Implementations, 2005.

\bibitem{Condat-2013}L. Condat. \emph{A Direct Algorithm for 1D Total
Variation Denoising}, IEEE Signal Proc. Letters, Vol. 20, No. 11,
pp. 1054-1057, 2013.

\bibitem{He-2012}B. He, X. Yuan. \emph{On the $O(1/n)$ convergence
rate of the Douglas-Rachford alternating direction method}, SIAM J.
Numer. Anal., Vol. 50, No. 2, pp. 700-709, 2012.

\bibitem{Berisha-2014}S. Berisha, J.G. Nagy. \emph{Iterative methods
for image restoration}, Academic Press Library in Signal Processing,
Volume 4, pp. 193-247, 2014.

\bibitem{Chuan-2014}H. Chuan, H. Chuang-Hua, Z. Wei, S. Biao. \emph{Box-constrained
Total-variation Image Restoration with Automatic Parameter Estimation},
Acta Automatica Sinica, Vol. 40 No.8, August 2014.

\bibitem{Deledalle-2012}C. Deledalle, S. Vaiter, G. Peyr�, J.M. Fadili,
C. Dossal. \emph{Unbiased Risk Estimation for Sparse Analysis Regularization},
Proc. ICIP\textquoteright 12, Sep 2012.
\end{thebibliography}

\end{document}
